

\begin{frame}[plain]{Optimization Strategies}{Troubleshooting}
\large
\pause
	First tests: Not so good...\\
	\pause
	Why does the car lose the line?\\
	\pause
	Use laptop instead of Raspberry Pi: 
		\pause
		\begin{itemize}
			\large
			\item Car worked perfectly fine
			\pause
			\item Reason: 
			\pause
			\begin{itemize}
				\large
				\item Computation power
				\pause
				\item $\approx$ 4ms per frame \pause (Pi: $\approx$170ms)
				\pause
				\item Camera: 15 FPS \pause $\Rightarrow$ new frame every 66ms
			\end{itemize}	
			\pause
			\item BUT: Algorithm concept works!
			
		\end{itemize}		 
\end{frame}
 
  
\begin{frame}[plain]{Optimization Strategies}{Guideline}
\large
	Main Problems:
	\pause
	\begin{itemize}
	\large
		\item Runtime
		\begin{itemize}
		\large
			\item Reaction time $\Leftrightarrow$ feasible speed
		\end{itemize}
		\pause
		\item Not all frames of the camera are processed
	\end{itemize}
	\pause
	\vspace{1.5cm}
	Main Improvements:
	\pause
	\begin{itemize}
	\large
		\item Improve runtime
		\begin{itemize}
		\large
			\item Improve reaction time $\Leftrightarrow$ feasible speed
		\end{itemize}
		\pause
		\item Process every frame
		\begin{itemize}
		\large
			\item Improve feasible speed
		\end{itemize}
	\end{itemize}
	
\end{frame}

\begin{frame}[plain]{Optimization Strategies}{Optimize Runtime}
\large
\pause
Disabled Gaussian Filter
\pause
\begin{itemize}
	\large
	\item Saves a lot of computation time
	\pause
	\item Replaced it by simple hardware trick
	\pause
	\begin{itemize}
		\large
		\item changed focus of camera \pause $\Rightarrow$ image gets blurred
	\end{itemize}
\end{itemize}
\pause
Restrict search range 
\pause
\begin{itemize}
	\large
	\item Before: search total line for largest Gradient
	\pause
	\item Now: only search at x-Position of last line $\pm$80 pixels
\end{itemize}
\pause
Only consider Gradients >25
\pause
\begin{itemize}
	\large
	\item If not found in search range: ignore line
\end{itemize}
\end{frame}

\begin{frame}[plain]{Optimization Strategies}{exploit concurrency}
\large
\pause
	Optimized runtime: $\approx$100-110ms per frame
	\pause
	\begin{itemize}
		\large
		\item Pretty good, but still every second frame lost
		\pause
		\item We are only using one core\pause, but Raspberry Pi has 4!
	\end{itemize}
	\pause
	\vspace{.5cm}
	Recompile OpenCV WITH\_OPENMP
	\pause
	\begin{itemize}
		\large
		\item Uses OpenMP to parallelize OpenCV calls
		\pause
		\item Runtime: $\approx$80ms 
		\pause
		\item still >66ms \pause $\Rightarrow$ Not good enough
	\end{itemize}
	
\end{frame}

\begin{frame}[plain]{Optimization Strategies}{exploit concurrency}
\pause
%\large
If we can't catch all frames with one thread why not use two?
\vspace{.5cm}
\begin{itemize}
	\pause
	\item Main thread starts two worker threads and sleeps
	\pause
	\item Worker thread grabs a image from the camera (synchronized)
	\pause
	\item Worker processes the image 
	\pause
	\item Worker checks whether a later frame had already finished
	\pause
	\item Worker wakes Main thread and communicates the calculated direction
	\pause
	\item Main thread generates UART signal and sleeps again
\pause
\end{itemize}
\vspace{.5cm}
This does not affect Reaction time, but increases performance
\end{frame}

\begin{frame}[plain]{Optimization Strategies}{exploit concurrency}
\pause
	Our approach and the WITH\_OPENMP option don't work well together
	\pause
		\begin{itemize}
			\item Runtime $\approx$120-130ms per frame (using 2 threads)
			\pause
			\item Probably to many threads started for the Pi
		\end{itemize}
		\pause
	Without using multithreading in OpenCV
		\begin{itemize}
		\pause
			\item Runtime $\approx$100-110ms per frame (using 2 threads)
			\pause
			\item Threads grab frames alternating
			\pause
			\item Every frame gets processed
		\end{itemize}
		
\end{frame}

\begin{frame}[plain]{Optimization Strategies}{summary}
	\large
	\pause
	Optimizations:
	\begin{itemize}
		\large
		\pause
		\item Reduce runtime per frame
		\pause
		\item Use multiple threads
		\pause
		\item Ensure every frame of the camera is used
	\end{itemize}
	\pause
	\vspace{1cm}
	Result: 
	\begin{itemize}
	\pause
		\item Grate performance is achieved
		\pause
		\item Car follows the line
		\pause
		\item Increase Speed of the car
	\end{itemize}			
\end{frame}